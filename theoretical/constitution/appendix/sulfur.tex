\chapter{硫黄}

\section{概要}{
	  {\bf 硫黄}(sulur) は 原子番号 $16$, 原子量 $32.1$ の元素である.
	元素記号は$S$. 酸素族元素の一つ. 多くの同素体や結晶多形が存在し、融点、密度はそれぞれ異なる。
	沸点は$444.674 {}^\circ\mathrm{C}$. 非金属.
}

\section{用途}{
	  硫黄から生成される{\bf 硫酸}は化学工業上、最も重要な{\bf 酸}である。一般に酸として用いられるのは希硫酸、脱水剤や乾燥剤に用いられるのは濃硫酸である。
	また、種々の硫黄を含んだ化合物が合成されている。\\
	\\
	硫黄は黒色火薬の原料であり、合成繊維、医薬品や農薬、また抜染剤などの重要な原料であり、様々な分野で硫化物や各種の化合物が構成されている。
	農家における干し柿、干しイチジクなどの漂白剤には、硫黄を燃やすことで得られる{\bf 二酸化硫黄}が用いられる(燻蒸して行われる)。 \\
	\\
	ゴムに数\%の硫黄を加えて加熱すると、{\bf 架橋}により、弾性がまし、更に添加量を増やすと硬さを増していき、最終的には{\bf エボナイト}となる。\\
	\\
	また、金属の硫化鉱物は{\bf 半導体}の性質を示すものが多く、シリコン鉱石検波器やゲルマニウムダイオードが実用化される以前は、鉱石検波器の主要部品として重用された。
}

\section{同素体}{
	硫黄は$30$以上の同素体を形成するが、これは他の元素に比べてもかなり多い。通常、天然にみられる同素体は環状の$S_8$である。\\
	常温、常圧で固体である$S_8$硫黄は3つの結晶系を持つ。\\
	\\
	\begin{description}
		\item[α硫黄(斜方硫黄)] - 融点 $112.8 {}^\circ\mathrm{C}$、比重$2.07$、淡黄色斜方晶
		\item[β硫黄(単車硫黄)] - 融点 $119.6 {}^\circ\mathrm{C}$、比重$1.96$、淡黄色単斜晶
		\item[γ硫黄(単車硫黄)] - 融点 $106.8 {}^\circ\mathrm{C}$、比重$1.955$、淡黄色針状晶
	\end{description}

	いずれも、$S_8$硫黄を単位構造とする結晶であるが、$95.6 {}^\circ\mathrm{C}$以下では斜方硫黄が安定であり、それ以上の温度では単斜硫黄系が安定である。また、$250  {}^\circ\mathrm{C}$まで加熱すると$50$万個の硫黄原子がつながった直鎖状硫黄$S_n$となる。これは、ゴム状硫黄またはプラスチック硫黄とも呼ばれる。
}

\section{性質}{
	  $S_8$硫黄は融点直上の温度では黄色をしており、粘性も低いが、温度が上昇するにつれて直鎖状硫黄へと変化が進み、$159.4 {}^\circ\mathrm{C}$以上では暗赤色となり年生が増大し殆ど流動性を失う。この温度以上では、$S_8$硫黄の感が解裂し、直鎖状のビラジカルが発生し、直鎖状$S_{16}, S_{24}$などのオリゴマー化が進行し直鎖状硫黄($S_n$)が形成され年生が急激に増大する。
}