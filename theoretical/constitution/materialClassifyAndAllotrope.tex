\chapter{物質の構成, 同素体}

\section{元素}{
	物質を構成する {\bf 原子} の種類. {\bf 元素記号}を用いて表す.
	Ex: $H, He, Li$
}

\section{物質}{
	物質は {\bf 純物質} と {\bf 混合物} の2つに分けることができる. \\
	\subsection{純物質}{
		{\bf 1種類の物質}から構成される(一つの化学式で表せる)物質. \\
		また,純物質は {\bf 単体} と {\bf 化合物} の2つに分類することができる. \\
		\subsubsection{単体}{
			{\bf 1種類の元素} から構成される物質. (Ex: $H_2, Na$)
		}
		\subsubsection{化合物}{
			{\bf 複数の元素} から構成される物質. (Ex: $H_2O, CO_2$)
		}
	}
	\subsection{混合物}{
		{\bf 複数の純物質}が混合された(一つの化学式で表せない)物質. (Ex: 空気, 食塩水)
	}
	
	{\bf 純物質は融点・沸点が物質によって一定の値をしめす}が,{\bf 混合物はその混合割合によって融点・沸点が変化}する. \\
	したがって,純物質であるか混合物であるかは,融点・沸点を調べることで判断ができる.
	
	\subsection{参考(空気の組成)}{
		\subsubsection{空気の組成(体積比)}{
			$N_2 : 78\%, O_2 : 21\%, Ar : 0.9\%, CO_2 : 0.04\% $ \\
			(Wikipediaより: \url{https://ja.wikipedia.org/wiki/空気})
		}
	}
}

\section{同素体}{
	{\bf 同じ元素から構成される単体}で,{\bf 性質が互いに異なるもの} \\

	\begin{table}[htp]
		\caption{同素体}
			\begin{center}
				\begin{tabular}{|c||l|}
					\hline
					硫黄 $S$ & 斜方硫黄$S_8$, 単斜硫黄$S_8$, ゴム状硫黄$S_x$ \\ \hline
					硫黄 $C$ & ダイヤモンド$C$, 黒鉛$C$, フラーレン$C_{60}$ \\ \hline
					酸素 $O$ & 酸素$O_2$, オゾン$O_3$ \\ \hline
					リン $P$ & 黄リン$P_4$, 赤リン$P_x$ \\ \hline
				\end{tabular}
			\end{center}
		\label{default}
	\end{table}%

	語呂合わせ: {\bf 同素体はSCOP(スコップ)で掘れ!}
}


\section{元素と単体の区別}{
	元素と単体は同じ名称で呼ばれるため,どちらの意味で用いられているのかに注意が必要である. (国語的な問題) \\
	{\bf 元素は「原子の種類」}, {\bf 単体は「実在する物質」}を意味する. \\
	以下に具体的な例を示す.
	\\

	\begin{itemize}
		\item 単体$O_2$や単体$O3$は, 1種類の元素$O$で構成される. \\
		\item 化合物$H_2O$は2種類の元素($H$と$O$)で構成される. (原子は$3$個) \\
		\item 化合物$H_2O$を電気分解すると, 単体$H_2$と単体$O_2$に分解される. \\
		\item 近く中には多量の元素$O$が存在する. \\
		\item 元素$Ca$は人にとって必要な栄養素である.
	\end{itemize}
}

\section{SCOPについて}{
	先程のSCOPの同位体について, それぞれ次からの章で説明する.
	\chapter{硫黄}

\section{概要}{
	  {\bf 硫黄}(sulur) は 原子番号 $16$, 原子量 $32.1$ の元素である.
	元素記号は$S$. 酸素族元素の一つ. 多くの同素体や結晶多形が存在し、融点、密度はそれぞれ異なる。
	沸点は$444.674 {}^\circ\mathrm{C}$. 非金属.
}

\section{用途}{
	  硫黄から生成される{\bf 硫酸}は化学工業上、最も重要な{\bf 酸}である。一般に酸として用いられるのは希硫酸、脱水剤や乾燥剤に用いられるのは濃硫酸である。
	また、種々の硫黄を含んだ化合物が合成されている。\\
	\\
	硫黄は黒色火薬の原料であり、合成繊維、医薬品や農薬、また抜染剤などの重要な原料であり、様々な分野で硫化物や各種の化合物が構成されている。
	農家における干し柿、干しイチジクなどの漂白剤には、硫黄を燃やすことで得られる{\bf 二酸化硫黄}が用いられる(燻蒸して行われる)。 \\
	\\
	ゴムに数\%の硫黄を加えて加熱すると、{\bf 架橋}により、弾性がまし、更に添加量を増やすと硬さを増していき、最終的には{\bf エボナイト}となる。\\
	\\
	また、金属の硫化鉱物は{\bf 半導体}の性質を示すものが多く、シリコン鉱石検波器やゲルマニウムダイオードが実用化される以前は、鉱石検波器の主要部品として重用された。
}

\section{同素体}{
	硫黄は$30$以上の同素体を形成するが、これは他の元素に比べてもかなり多い。通常、天然にみられる同素体は環状の$S_8$である。\\
	常温、常圧で固体である$S_8$硫黄は3つの結晶系を持つ。\\
	\\
	\begin{description}
		\item[α硫黄(斜方硫黄)] - 融点 $112.8 {}^\circ\mathrm{C}$、比重$2.07$、淡黄色斜方晶
		\item[β硫黄(単車硫黄)] - 融点 $119.6 {}^\circ\mathrm{C}$、比重$1.96$、淡黄色単斜晶
		\item[γ硫黄(単車硫黄)] - 融点 $106.8 {}^\circ\mathrm{C}$、比重$1.955$、淡黄色針状晶
	\end{description}

	いずれも、$S_8$硫黄を単位構造とする結晶であるが、$95.6 {}^\circ\mathrm{C}$以下では斜方硫黄が安定であり、それ以上の温度では単斜硫黄系が安定である。また、$250  {}^\circ\mathrm{C}$まで加熱すると$50$万個の硫黄原子がつながった直鎖状硫黄$S_n$となる。これは、ゴム状硫黄またはプラスチック硫黄とも呼ばれる。
}

\section{性質}{
	  $S_8$硫黄は融点直上の温度では黄色をしており、粘性も低いが、温度が上昇するにつれて直鎖状硫黄へと変化が進み、$159.4 {}^\circ\mathrm{C}$以上では暗赤色となり年生が増大し殆ど流動性を失う。この温度以上では、$S_8$硫黄の感が解裂し、直鎖状のビラジカルが発生し、直鎖状$S_{16}, S_{24}$などのオリゴマー化が進行し直鎖状硫黄($S_n$)が形成され年生が急激に増大する。
}
}
